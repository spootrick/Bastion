\documentclass[12pt,a4paper]{article}
\usepackage[margin=2cm,bottom=2cm]{geometry}
\usepackage[utf8]{inputenc}
\usepackage{hyperref}
\usepackage{graphicx}
\usepackage[table]{xcolor}
%page anchorlar gerekirse alt satırı sil
\hypersetup{pageanchor=false}

\begin{document}
\begin{titlepage}
 \centering
 \includegraphics[scale=0.3]{bilgi_logo}\\
 {\scshape\LARGE Istanbul Bilgi University\\}
 \vspace{1cm}
 {\scshape\Large Laureate Bilgi Robotic Competition\\}
 \vspace{2cm}
 {\huge\bfseries BASTION THE SENTINEL\\}
 {\scshape\Large Park Cleaner Robot\\}
 \vspace{2cm}
 {\Large\itshape Ömer Faruk Bakırcı\\}
 {\Large\itshape Tolga Akdemir\\}
 {\Large\itshape Şeyhmus Okan Pordoğan\\}
 {\Large\itshape Furkan Karakoyunlu\\}
 \vspace{4cm}
 {\Large\itshape Supervisor\\Eray Baran}
 \vfill
 \vfill
 {\large \today\\}
\end{titlepage}


% Table of contents
\tableofcontents
\pagebreak

\section{Scope}
 \begin{flushleft}
  Today's era consuming is essential for human beings, producing garbage every minute and it leaves great amount of damage 
  to the nature. How many of people recycle? The answer is not so much. Thus, there is huge space to filled for recycling 
  and \textbf{Bastion The Sentinel} will help to close the gap. We are aiming for people do not waste their time for garbage 
  collecting and separating for recycling from the beginning. The purpose of creating this robot is reducing manpower in 
  garbage collecting and increasing recycling cycle for the make future better.
 \end{flushleft}
  
\section{Device Functionality}
 \begin{flushleft}
  \textbf{Bastion The Sentinel} is designed as a semi-autonomous robot. It will be a multi-tasking robot, in addition to 
  image processing abilities it can also be controlled by operator. In the image processing part first it will consider color, 
  if it fails \textbf{(olumlu bi cümle yazmamız lazım)} perform image processing for the shape of the object. It will separate 
  all of the garbage by particularly and put them into bin which is placed on top of the robot in separate places. For increasing 
  the maneuver ability, we placed steering wheels to the back, besides we choose to place impulse to the front of the robot. The 
  reason for choosing this method is while its collecting the garbages on the back, it will increase the weight because of that 
  putting impulse in front will be more reasonable. Another significant thing is deciding the sensors. On behalf of understand 
  approaching an object we will use proximity sensor. In addition to proximity sensor we will use depth sensor for calculating 
  the depth. \textbf{color sensör yazmadık, depth, proximity ve kamera birlikte nasıl çalışacak}
 \end{flushleft}
 
 \section{Overall Design Scheme}
 \begin{flushleft}
  The block diagram which represents the overall system design is shown in figure below\\
  buraya resim gelecek\\
  The overall system of \textbf{Bastion The Sentinel} consists of 4 main blocks. There are ..(blok isimleri)
 \end{flushleft}
 
 \section{Details of Design}
  
   \subsection{Input Block}
    \begin{flushleft}
     \textbf{Bastion The Sentinel} will collect data from environment via its sensors and cameras. These data will be used to 
     feed micro controllers. The main points are:
     \begin{itemize}
      \item \textbf{Bastion The Sentinel}'s robotic arm divided by three joints and has one gripper.
      \item Robotic arm can be rotated by 360$^{\circ}$ and it can be collapsed to minimize the area. Also the gripper part will 
      have the ability to turn up to 180$^{\circ}$.
      \item Garbage detection is reinforced by proximity sensor. It will provide information about whether there is an object around 
      of the robot.
      \item Depth sensor will be used to verify data's incoming from proximity sensor. If they match the robotic arm will start the 
      collecting process.
     \end{itemize}

    \end{flushleft}

    \subsubsection{Cameras}
     \begin{flushleft}
      
     \end{flushleft}
   \subsection{Control Block}
    \begin{flushleft}
     buraya bi resim gelecek çpe yaklaşımı ve duracağı mesafeyi göstericez\\
     In this block, Arduino (and Raspberry Pi) will manage the wheel system. It gives impulse to front wheels while it only 
     rotates back wheels. (burayı biraz uzatabiliriz)
    \end{flushleft}

   \subsection{Output Block}
    \begin{flushleft}
     Micro controllers empowers the DC motors for movement process on front wheels and also it empowers servo motors for 
     rotating back wheels. After the data has been collected from the environment, micro controllers uses the robotic arm 
     to gather objects to its container. All this information will be shared with the operator.\\
     \textbf{Bastion The Sentinel} can use artificial intelligence to determine and separate the objects.
    \end{flushleft}

   \subsection{Communication Block}
    \begin{flushleft}
     burayı kenana sorduktan sonra yazıcaz
    \end{flushleft}

 \section{Time Table and Work Schedule}
  \begin{flushleft}
   
  \end{flushleft}
  
 \section{CAD Design, Device Dimensions and Weight}
  \begin{flushleft}
   
  \end{flushleft}

 \section{Budget}
  \subsection{Budget Breakdown}
  \begin{flushleft}
   % colors of the tabular
   \definecolor{Gray}{gray}{0.95}
   \definecolor{LightCyan}{rgb}{0.88,1,1}
   \definecolor{LightGreen}{rgb}{0.88,0.9,0.9}
   
   \newcolumntype{g}{>{\columncolor{Gray}}c}
   
   \begin{tabular}{ |c|g|c|g|c| }
    \hline
    \rowcolor{LightGreen}
    \multicolumn{5}{ |c| }{Materials}                                    \\
    \hline
    \rowcolor{LightCyan}
    Number &   Material Name   & Quantity & Unit Price (\$) & Total Cost(\$) \\
    \hline
           & DC Motor          & 2        & \$30            & \$60       \\
    \hline
           & Servo Motor       & 6        & \$8             & \$48       \\
    \hline
           & Wheel             & 4        & \$15            & \$60       \\
    \hline 
           & Bearing Hub       & 4        & \$12            & \$48       \\
    \hline
           & Carbon Fiber Chassis  & 1    & \$60            & \$60       \\
    \hline
           & Upper Plexy Body  & 1        & \$45            & \$45       \\
    \hline
           & Robotic Arm Components & 1   & \$250           & \$250      \\
    \hline
    \rowcolor{LightGreen}
    \multicolumn{5}{ |c| }{Kits}                                         \\
    \hline
    \rowcolor{LightCyan}
    -&   Kit Name        & Quantity & Unit Price (\$) & Total Cost(\$)   \\
    \hline
     & Raspberry Pi3 Maxi & 1        & \$150            & \$150          \\
    \hline
     & Arduino Mega Rev3 & 1        & \$40            & \$40             \\
    \hline
     & Arduino Motor Shield & 2     & \$15            & \$30             \\
    \hline
    \rowcolor{LightGreen}
    \multicolumn{5}{ |c| }{Sensors}                                      \\
    \hline
    \rowcolor{LightCyan}
    -&   Sensor Name        & Quantity & Unit Price (\$) & Total Cost(\$)  \\
    \hline
    & Proximity Sensor      & 1        & **            &           \\
    \hline
    & Depth Sensor      & 1        & \$            & \$          \\
    \hline
    & Light Sensor      & 1        & **            &           \\
    \hline
    & CMOS Camera Sensor & 2       & \$35          & \$70           \\
    \hline
    \rowcolor{LightGreen}
    \multicolumn{5}{ |c| }{Communication and Internet}                 \\
    \hline
    \rowcolor{LightCyan}
    -&   Name        & Quantity & Unit Price (\$) & Total Cost(\$)  \\
    \hline
    & Wireless       & 1        & **              &                    \\
    \hline
    & Domain Name    & 1        & \$10            & \$10            \\
    \hline
    \rowcolor{LightGreen}
    \multicolumn{5}{ |c| }{Miscellaneous}                         \\
    \hline
    \rowcolor{LightCyan}
    -&   Name        & Quantity & Unit Price (\$) & Total Cost(\$)  \\
    \hline
    & 240 Volt FuckMaster Pro 5000 & 3 & \$1000 & \$3000 \\
    \hline
   \end{tabular}
   \\ 
   \footnotesize \textit{** indicates Raspberry Pi3 Maxi Kit includes this sensor }\normalsize
  \end{flushleft}

  \subsection{Total Budget}
   \begin{flushleft}
    The prices for the components of the \textbf{Bastion The Sentinel} are shown in the table.
    \begin{itemize}
     \item Base price: \$
     \item Worst case scenerio: \$
     \item Total: \$
    \end{itemize}

   \end{flushleft}

  
 \section{Safety and Security Features}
  \begin{flushleft}
   
  \end{flushleft}

 \section{Business Plan}
 \begin{flushleft}
  
 \end{flushleft}


\end{document}

