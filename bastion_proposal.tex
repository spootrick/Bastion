\documentclass[12pt,a4paper]{article}
\usepackage[margin=2cm,bottom=2cm]{geometry}
\usepackage[utf8]{inputenc}
\usepackage{hyperref}
\usepackage{graphicx}
%page anchorlar gerekirse alt satırı sil
\hypersetup{pageanchor=false}

\begin{document}
\begin{titlepage}
 \centering
 \includegraphics[scale=0.3]{bilgi_logo}\\
 {\scshape\LARGE Istanbul Bilgi University\\}
 \vspace{1cm}
 {\scshape\Large Laureate Bilgi Robotic Competition\\}
 \vspace{2cm}
 {\huge\bfseries BASTION THE SENTINEL\\}
 {\scshape\Large Park Cleaner Robot\\}
 \vspace{2cm}
 {\Large\itshape Ömer Faruk Bakırcı\\}
 {\Large\itshape Tolga Akdemir\\}
 {\Large\itshape Şeyhmus Okan Pordoğan\\}
 {\Large\itshape Furkan Karakoyunlu\\}
 \vspace{4cm}
 {\Large\itshape Supervisor\\Eray Baran}
 \vfill
 \vfill
 {\large \today\\}
\end{titlepage}


% Table of contents
\tableofcontents
\pagebreak

\section{Scope}
 \begin{flushleft}
  Today's era consuming is essential for human beings, producing garbage every minute and it leaves great amount of damage 
  to the nature. How many of people recycle? The answer is not so much. Thus, there is huge space to filled for recycling 
  and \textbf{Bastion The Sentinel} will help to close the gap. We are aiming for people do not waste their time for garbage 
  collecting and separating for recycling from the beginning. The purpose of creating this robot is reducing manpower in 
  garbage collecting and increasing recycling cycle for the make future better.
 \end{flushleft}
  
\section{Device Functionality}
 \begin{flushleft}
  \textbf{Bastion The Sentinel} is designed as a semi-autonomous robot. It will be a multi-tasking robot, in addition to 
  image processing abilities it can also be controlled by operator. In the image processing part first it will consider color, 
  if it fails \textbf{(olumlu bi cümle yazmamız lazım)} perform image processing for the shape of the object. It will separate 
  all of the garbage by particularly and put them into bin which is placed on top of the robot in separate places. For increasing 
  the maneuver ability, we placed steering wheels to the back, besides we choose to place impulse to the front of the robot. The 
  reason for choosing this method is while its collecting the garbages on the back, it will increase the weight because of that 
  putting impulse in front will be more reasonable. Another significant thing is deciding the censors. On behalf of understand 
  approaching an object we will use proximity censor. In addition to proximity censor we will use depth censor for calculating 
  the depth. \textbf{color sensör yazmadık, depth, proximity ve kamera birlikte nasıl çalışacak}
 \end{flushleft}
 
 \section{Overall Design Scheme}
 \begin{flushleft}
  The block diagram which represents the overall system design is shown in figure below\\
  buraya resim gelecek\\
  The overall system of \textbf{Bastion The Sentinel} consists of 4 main blocks. There are ..(blok isimleri)
 \end{flushleft}
 
 \section{Details of Design}
  \begin{flushleft}
   \subsection{Input Block}
    \begin{flushleft}
     \textbf{Bastion The Sentinel} will collect data from environment via its censors and cameras. These data will be used to 
     feed micro controllers. The main points are:
     \begin{itemize}
      \item \textbf{Bastion The Sentinel}'s robotic arm divided by three joints and has one gripper.
      \item Robotic arm can be rotated by 360$^{\circ}$ and it can be collapsed to minimize the area. Also the gripper part will 
      have the ability to turn up to 180$^{\circ}$.
      \item Garbage detection is reinforced by proximity censor. It will provide information about whether there is an object around 
      of the robot.
      \item Depth censor will be used to verify data's incoming from proximity censor. If they match the robotic arm will start the 
      collecting process.
     \end{itemize}

    \end{flushleft}

    \subsubsection{Cameras}
     \begin{flushleft}
      
     \end{flushleft}
   \subsection{Control Block}
   \subsection{Output Block}
   \subsection{Communication Block}
  
 \end{flushleft}
 
 \section{Time Table and Work Schedule}
  \begin{flushleft}
   
  \end{flushleft}
  
 \section{CAD Design, Device Dimensions and Weight}
  \begin{flushleft}
   
  \end{flushleft}

 \section{Budget}
  \subsection{Budget Breakdown}
  \subsection{Total Budget}
  
  
 \section{Safety and Security Features}
  \begin{flushleft}
   
  \end{flushleft}

 \section{Business Plan}
 \begin{flushleft}
  
 \end{flushleft}


\end{document}

